\ifx\allfiles\undefined
\documentclass[12pt]{book}
\usepackage[a4paper]{geometry}
\usepackage[dvipsnames]{xcolor}
\usepackage{amsmath, amsthm, amssymb, mathrsfs}
\usepackage{graphicx}
\usepackage{enumitem}

\usepackage{tikz}
\usepackage[colorlinks,linkcolor=black]{hyperref}
%\usepackage{biblatex}

\pagestyle{plain}

\graphicspath{ {fig/},{../fig/}, {config/}, {../config/} }

\geometry{top=25.4mm,bottom=25.4mm,left=20mm,right=20mm,headheight=2.17cm,headsep=4mm,footskip=12mm}
\linespread{1.5}
\setenumerate[1]{itemsep=5pt,partopsep=0pt,parsep=\parskip,topsep=5pt}
\setitemize[1]{itemsep=5pt,partopsep=0pt,parsep=\parskip,topsep=5pt}
\setdescription{itemsep=5pt,partopsep=0pt,parsep=\parskip,topsep=5pt}

% Theorem environment
\newtheorem{defn}{Definition}[section]

\newtheorem{lemma}{Lemma}[section]
\newtheorem{theorem}[lemma]{Theorem}
\newtheorem{corollary}[lemma]{Corollary}
\newtheorem{criterion}[lemma]{Criterion}

\newtheorem{proposition}{Proposition}[section]
\newtheorem{example}{Example}[section]
\newtheorem*{rmk}{Remark}


\def\d{\textup{d}}
\begin{document}
% \title{{\Huge Lecture-Notes on \\ \textbf{
Computational Methods in Evolutionary Biology}}}
\author{Chang Longxiao\\
longxiao.chang@campus.lmu.de}
\date{\today}

\maketitle                   % title in a newpage

\thispagestyle{empty}        % without page number
\begin{center}
    \Huge\textbf{FOREWORD}
\end{center}

"\textit{Nothing in biology makes sense except in the light of evolution}," said Theodosius Dobzhansky. Likewise, we could say that \textit{nothing in evolution makes sense except in the light of mathematics}.

As a child, I was fascinated by the vivid beauty of nature. As I grew older, this fascination led me to biology—the science that seeks to describe the dynamics of the living world. However, such descriptions often feel too imprecise, if not unscientific, to truly grasp the complexities of reality. Contrary to what many experimental biologists believe, There is too less mathematic languages used in biology, not too much.
%biology does not suffer from an excess of mathematics, but rather from a shortage of it.

Evolution, as the most fundamental and important topic in biology, appears to have the clearest structure that can be described mathematically, largely due to its long time scale. This note serves as a record of the models and methods I have learned in evolutionary biology. By building upon previous research, we may one day uncover a deeper understanding of the intricate complexity of life.

\begin{flushright}
    \begin{tabular}{c}
        \today \\ 
        Im Biozentrum LMU
    \end{tabular}
\end{flushright}

\iffalse
\begin{center}
    If people do not believe that mathematics is simple,

    it is only because they do not realize how complicated life is. 
    
    ——John von Neumann
\end{center}
\fi

\newpage                   
\pagestyle{plain}             
\setcounter{page}{1}          
\pagenumbering{Roman}    

\tableofcontents              

\newpage                   
\pagestyle{plain}
\setcounter{page}{1}          
\pagenumbering{arabic} 
\else
\fi

\setcounter{chapter}{-1}    %introduction from chapter 0

\chapter{Introduction}
%  content
\section{Outline}
    \subsection{Phylogenetics part}
        \begin{itemize}
            \item Tree Notation
            \item Distance-Based and Maximum Parsimony Phylogeny Reconstruction
            \item Measures for how different two trees are
            \item Sequence Evolution Models (JC, F81, HKY, F84, GTR, PAM and $\Gamma$-distributed rates)
            \item Maximum-Likelihood (ML) in phylogeny estimation and Bootstrapping
            \item Bayesian phylogeny reconstruction and MCMC
            \item Common problems in phylogenetics and consequences for phylogenomics
            \item Modelling the substitution process on sequences
            \item Quantitative Characters and Independent Contrasts
            \item Model selection
            \item Statistical Alignment (TKF91, TKF92, pairHMMs, multiple HMMs)
            \item Tests for trees and branches
        \end{itemize}

    \subsection{Population Genetics part}
        \begin{itemize}
            \item Wright-Fisher model and Estimators of $\theta$
            \item population genealogy with Mutation and Immigration
            \item Important Sampling and Approximate Bayesian Computation (ABC)
            \item Recombination: ARG and approximations
            \item Selection: ASG and detection
            \item population cluster
        \end{itemize}

    \subsection{Related mathematics}
        \begin{itemize}
            \item Basic concepts from probability theory
            \item Basic linear algebra
            \item Markov process and Markov chain
            \item \dots
        \end{itemize}
    
    \subsection{Aims}
        \begin{itemize}
            \item Understand princples and rationales underlying the methods
            \item Explore available software
            \item What is efficiently doable, what is difficult?
            \item What are the strengths and weaknesses of the methods?
            \item Which method is appropriate for which dataset?
            \item Learn what is necessary to read papers about new computational methods
            \item Future directions of phylogenetics
        \end{itemize}

    \subsection{Recommended Books}\hfill
        
        J. Felsenstein (2004) Inferring Phylogenies
        
        Z. Yang (2006) Computational Molecular Evolution
        
        R. Nielsen, [Ed.] (2005) Statistical Methods in Molecular Evolution
        
        R. Durbin, S. Eddy, A. Krogh, G. Mitchison (1998) Biological Sequence Analysis
        
        W. Ewens, G. Grant (2005) Statistical Methods in Bioinformatics

%  content
\ifx\allfiles\undefined
\end{document}
\fi